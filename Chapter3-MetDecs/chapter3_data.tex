\section{Data}
\label{sec:data}

This work uses integral-field spectroscopic (IFS) data from the MaNGA survey \citep{bundy15_manga}, part of SDSS-IV \citep{blanton_17_sdss-iv}. MaNGA will observe more than 10,000 nearby galaxies ($0.01 < z < 0.15$) from the NASA-Sloan Atlas \citep[NSA, ][]{blanton_11_nsa}, with an approximately-uniform distribution in $i$-band absolute magnitude, resulting in a roughly-flat distribution in $\log M^*$, and is approximately volume-limited within a given redshift range \citep{manga_sample_wake_17}. Two-thirds of observed galaxies are drawn from the ``Primary+" sample (coverage to at least 1.5 $R_e$); and the remainder come from the ``Secondary" sample (covered to at least 2.5 $R_e$). The spectroscopic data used in this study come from \mplvfull (\mplv), an internally-released dataset nearly identical to SDSS Data Release 16 \citep{sdss_dr16}, and containing \mplvngal galaxies.

MaNGA spectroscopic observations cover the wavelength range of 3600 to 10300 $\mbox{\AA}$ with $d\log\lambda \sim 10^{-4}$ ($R \sim 2000$), and use the BOSS spectrograph \citep{smee_boss_instrument, sdss_boss_dawson_13}, an instrument at the SDSS 2.5-meter telescope at Apache Point Observatory \citep{gunn_sdss_telescope}. To achieve uniform spatial sampling, the spectrograph is coupled to closely-packed fiber hexabundles, each containing between 19 and 127 fibers \citep{manga_inst}. Sky-subtraction and spectrophotometric calibration are accomplished using single fibers and smaller hexabundles \citep{manga_drp, manga_spectrophot}. All hexabundles and sky fibers are inserted into a plugplate affixed to the focal plane \citep{sdss_summary}. Sets of three ``dithered" pointings compose the MaNGA observations, and to form the datacubes (\texttt{CUBE} products), these observations are rectified to a spatial grid in the plane of the sky, with spaxel size 0.5" by 0.5" and seeing-induced PSF having an $i$-band FWHM $\sim 2.5$" \citep{manga_obs, manga_progress_yan_16, manga_drp}. The MaNGA Data Analysis Pipeline \citep[DAP, ][]{manga_dap} measures stellar kinematics, emission-line fluxes \citep{belfiore_2019_mangadap-emissionline}, and stellar spectral indices for individual spaxels.

This work builds on the results of two MaNGA Value-Added Catalogs (VACS). First, we use estimates of total galaxy stellar-masses from the MaNGA-PCA project, which used an orthogonal spectral basis set trained on realistic SFHs to obtain robust resolved galaxy stellar masses \citep{pace_19a_pca}; the resolved masses were then aperture-corrected to form a catalog of total stellar-masses \citep{pace_19b_pca}. These stellar masses are likely much more reliable than those included in the MaNGA targeting catalog, since galaxy-averaged light is more apt to ``miss" stellar-mass in dusty environments and other low-flux regions \citep{zibetti_2009, sorba_sawicki_15, pace_19b_pca}. Second, the MaNGA PyMorph DR15 photometric catalog provides parameters obtained from S\'{e}rsic and S\'{e}rsic-Exponential fits to MaNGA galaxies' plane-of-sky surface-brightness profiles \citep{fischer_2019_pymorph}. This allows radial abundance trends to be computed with respect to the disk (a more fundamental unit of chemical evolution), rather than the disk plus the bulge.

Finally, we include single-dish atomic hydrogen (HI) mass measurements and upper-limits: the GASS \citep{catinella_2010_GASS} and ALFALFA \citep{haynes_2018_alfalfa} surveys form the small archival portion of the HI measurements. The majority of measurements come from the HI-MaNGA project \citep{masters_19_himanga, goddy_2020_gbtcal, stark_himanga}, an observational campaign carried out with the Green Bank Telescope. This program targets MaNGA galaxies at $cz < 15,000 ~ {\rm km s^{-1}}$, but regardless of their morphology, with an intended depth of 1.5 mJy at $10 ~ {\rm km s^{-1}}$ (after spectral smoothing), resulting in a stellar-mass distribution of targets peaking at $M^* \sim 10^{9.8} {\rm M_{\odot}}$ \citep[see Figure 1 of][]{masters_19_himanga}. GBT HI observations were translated into HI mass estimates, and in the case of non-detections, mass upper-limits were estimated using the observational noise and assuming a rotation-curve with of $200 {\rm km ~ s^{-1}}$. In total, 3413 MaNGA galaxies have measured HI masses or upper-limits.