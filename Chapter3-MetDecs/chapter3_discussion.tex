\section{Discussion and Conclusions}
\label{sec:disc}

Galaxies are widely understood to exchange gas with their surroundings. In this work, we relate the abundance patterns present within individual MaNGA star-forming galaxies to their total atomic gas content. We find
\begin{itemize}
    \item Galaxies with high H\textsc{i} masses relative to their total stellar-masses tend to possess strong gas-phase metallicity decrements between their innermost star-forming regions ($0.0-0.5~R_e$) and those at slightly larger radii ($1.25-1.75~R_e$). This is similar to an effect noted in a sample of DustPedia galaxies \citep{devis_2019_dustpedia}.
    \item Those same galaxies also tend to have relatively wide distributions ($> 0.05~{\rm dex}$) of gas-phase metallicity in the radial interval $1.25-1.75 ~ R_e$, compared with both peer galaxies of similar mass and a separate sample of star-forming galaxies at higher spatial resolution, but with metallicities obtained using the same strong-line calibration \citep{kreckel_2019_phangs_metgrads}.
    \item These effects are limited to relatively low-mass galaxies: galaxies with $M^* > 10^{10.7} M_{\odot}$ seem to lack both strong metallicity decrements and large metallicity dispersions within the interval $1.25-1.75 ~ R_e$. This indicates that inflows occur less frequently, or have a smaller overall impact. At low stellar mass, the correlations involving HI are not statistically significant at the 5\% level. There are fewer galaxies overall in the lowest-mass bin, and unlike the highest-mass bin, the magnitude of the correlations between \metdec and \metdisp do not diminish at low mass. More HI observations of low-mass galaxies may result in more robust correlations with HI.
    \item Galaxies with abnormally-large metallicity dispersion between 1.5 and 2.5 $R_e$ have on average $0.05-0.25 ~ {\rm dex}$ more HI when normalized by total stellar mass.
\end{itemize}

We attempt to explain the combination of effects by invoking azimuthally-asymetric, low- (but not zero-) -metallicity inflows with a strong atomic component. Indeed, \citet{pezzulli_fraternali_2015} used an analytic chemical-evolution framework to predict that inward flows of gas produce steeper-than-normal metallicity gradients. The gas-rich inflows explored in this work are observationally distinct from those brought about by galaxy mergers, which have been shown to \emph{flatten} abundance gradients, rather than steepen them \citep{rupke_2010_inflows_metgrad}. Since enhanced radial metallicity decrements are also associated in our sample with increases in the \emph{metallicity dispersion at fixed radius}, we conclude that any inflow ought to have a significant covering fraction---at least 10\%, in order to noticeably widen the metallicity distribution in one radial interval; but less than 100\%, a case in which \metdec would rise, but \metdisp would not. Under our basic model, the plausible range of accreted ISM metallicities implies a range of inflow dilution factors (the margin by which a \emph{local} gas reservoir's mass is enhanced) of 20\%-100\%. Though a higher-metallicity inflow universally implies a larger inflow for the same pre-inflow ambient and post-inflow observed metallicity, this effect becomes all but negligible at higher stellar masses (higher ambient metallicities).

As discussed by \citet{schaefer_19_accretion}, the presence of a nearby massive halo is associated with accreted gas that is more enriched. This means that even within one mass bin, the metallicity of accreted gas could vary by a factor of several. In other words, the same enhanced metallicity decrement would require a smaller inflow by mass for a satellite of a low-mass galaxy than for a satellite of a high-mass galaxy. Once MaNGA observations and HI follow-up are complete, it may be possible to modulate the assumed inflow metallicity based on environment. Additionally, galaxy mass itself may impact the inflows it experiences: \citet{muratov_2017_FIRE_cgm_metals} finds that a high-mass halo ($M^* \sim 10^{10.8} {\rm M_{\odot}}$) rarely sustains inflows that reach its interior ($< 0.25~R_{vir}$); whereas at lower halo mass ($M^* \sim 10^{9.3} {\rm M_{\odot}}$), inflows have a duty cycle of about 50\%! This agrees qualitatively with the observed dearth of evidence for inflows in the highest-mass bin in this work.

Though the dilution factors obtained according to this simple model imply moderate enhancements in local gas supply, there ought to be a connection to the reservoir of gas supplying the entire galaxy. With the assumed inflow metallicities, the HI mass associated with the dilution effect in strictly the radial interval $1.25-1.75 R_e$ is substantially smaller than the average difference in HI fractions between the high-metallicity-dispersion and low-metallicity-dispersion populations. Preserving the connection between local and global gas supply seems to require a gaseous component with relatively large radial scale-length. The GBT observations (having a FWHM of 8.8') are capable of detecting HI residing far away from the star-forming disk, and any diluting inflows incident upon a galaxy's disk could be understood as a small fraction of gaseous disk that extends outward to $10 R_e$ or beyond \citep{bigiel_blitz_12-gasprofile}.

Alternative explanations to the diluting-inflow hypothesis include intra-galaxy modulation of star formation efficiency (SFE): \citet{schaefer_19_ohno} reports radial variations in SFE of nearly an order of magnitude, but the degree of variance within single galaxies and at constant radius is at present unexplored \citep[see also][]{bigiel_leroy_08}. Galaxy regions with lower SFE may have lower metallicity than peer regions at similar radius. In order to explain the diversity of HI mass fraction at fixed stellar mass, the SFE would have to be coherently depressed across the entire galaxy for a significant portion of the age of the universe, which we consider an unlikely scenario.

\begin{table*}
\centering
\begin{tabular}{|c|c|c|c|c|c|c|c|c|}
(1) & (2) & (3) & (4) & (5) & (6) & (7) & (8) & (9) \\
\texttt{plateifu} & \texttt{mangaid} & $M_{HI}$ (or U.L.) & HI meas. type & \logmstar & $z$ & \hifrac & \metdisp & \metdec \\ \hline\hline
9501-6101 & 1-384726 & 9.899 & 0 & 9.807 & 0.03924 & 0.09194 & 0.146 & 0.139 \\ \hline
8452-12701 & 1-167678 & nan & 2 & 9.948 & 0.04017 & nan & 0.131 & 0.185 \\ \hline
8323-12704 & 1-405813 & nan & 2 & 9.688 & 0.03819 & nan & 0.121 & 0.175 \\ \hline
8156-12703 & 1-38894 & 9.974 & 0 & 9.815 & 0.04192 & 0.159 & 0.117 & 0.333 \\ \hline
7495-9101 & 12-129610 & 9.443 & 1 & 9.168 & 0.03235 & 0.275 & 0.115 & 0.188 \\ \hline
9506-12704 & 1-299793 & 10.469 & 0 & 10.424 & 0.04884 & 0.0455 & 0.115 & 0.237 \\ \hline
9485-9102 & 1-121994 & nan & 2 & 8.980 & 0.01918 & nan & 0.114 & 0.237 \\ \hline
8146-9101 & 1-556506 & 9.819 & 0 & 9.538 & 0.02393 & 0.281 & 0.112 & 0.183 \\ \hline
8259-9101 & 1-257822 & 9.409 & 0 & 9.230 & 0.01980 & 0.180 & 0.101 & 0.169 \\ \hline
8936-6104 & 1-152828 & 9.419 & 0 & 9.361 & 0.01570 & 0.0584 & 0.0993 & -0.182 \\ \hline
\end{tabular}
\caption[A segment of the machine-readable table aggregating total galaxy stellar masses, chemical variations, and HI masses/upper-limits (where available)]{\fixspacing A segment of the machine-readable table aggregating total galaxy stellar masses, chemical variations, and HI masses/upper-limits (where available). \textbf{Columns (1) \& (2)} provide a galaxy's MaNGA-ID and \texttt{plate-ifu} designations, \textbf{column (3)} the HI mass or upper-limit (``nan" if not in the HI follow-up campaign), \textbf{column (4)} the HI measurement type (0 signifies a measurement, 1 an upper-limit, and 2 not targeted), \textbf{column (5)} the total galaxy stellar mass \citep{pace_19b_pca}, \textbf{column (6)} the optical redshift, \textbf{column (7)} the ratio of the HI mass to the stellar mass (\hifrac), \textbf{column (8)} the metallicity dispersion in the radial interval $1.25-1.75 ~ R_e$ (\metdisp), and \textbf{column (9)} the measured metallicity decrement (\metdec).}
\label{tab:galaxies}
\end{table*}

Like in \citet{hwang_2019_manga_almrs}, this study finds evidence for gaseous inflows affecting a sizeable fraction of star-forming galaxies in the nearby universe ($\sim 25\%$), as reflected in the presence of anomalously-low-metallicity, star-forming gas. While \citet{hwang_2019_manga_almrs} defines ALM gas according to a joint regression of metallicity against $M^*$ and $\Sigma^*$ (essentially, a global-local model), this study aims to detect the simple presence of ALM gas on a galaxy-by-galaxy basis in a galactocentric annulus between 1.25 \& 1.75 $R_e$ according to an elevated dispersion of the metallicity distribution function within that annulus, along with a steep metallicity decrement relative to the central value. This work also employs a different, likely more robust strong-line metallicity indicator (the \texttt{PG16-R2} indicator from \citealt{pilyugin_grebel_2016}); and takes extensive steps to minimize the contamination of gaseous emission-line signatures across radial scales, excluding a great many spaxels near galaxies' minor axes. We uncover a mutual link between a strong radial metallicity decrement (between the radial intervals $0-0.5 ~ R_e$ \& $1.25-1.75 ~ R_e$), elevated dispersion of the metallicity distribution function at $1.25-1.75 ~ R_e$, and total galaxy HI content; indicating a link between ALM gas and a galaxy's total gas reservoir. The elevated metallicity decrement itself is a symptom of a radial gas flow found in simple, analytic models of galaxy chemical evolution \citep{pezzulli_fraternali_2015}. The correlations between \metdec, \metdisp, and HI mass manifest at fixed total galaxy stellar-mass, up to a mass of $10^{10.7} ~ {\rm M_{\odot}}$, signaling that relatively low-mass galaxies constitute interesting laboratories for uncovering accretion signatures. We suggest targeting for high-resolution HI follow-up galaxies with wide metallicity distribution functions at fixed radius, pronounced metallicity gradients (or decrements), and large HI mass fractions for their stellar mass. A table is provided to aid in choosing targets for HI follow-up: a full version is included in machine-readable format, and we show a sample in Table \ref{tab:galaxies}. Resolved radio observations targeted according to chemistry may reveal coincident low-metallicity, star-formation-driven line emission \& cold gas enhancement, indicating active accretion from an external gas reservoir.