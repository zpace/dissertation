\section{Introduction}
\label{sec:intro}

% big picture

Galaxies must exchange gas with their immediate environments: the continuous consumption of gaseous reservoirs through star formation implies a need for replenishment \citep{kennicutt_evans_2012_sf-review}. For example, to sustain the Milky Way's current star formation rate (SFR) of $\sim 1.5 {\rm M_{\odot} yr^{-1}}$ over the past 8 Gyr (a factor of several greater than the depletion time, the reciprocal of the star formation efficiency), gas capable of forming stars should be introduced at a similar rate \citep{snaith_2015_mw-sfr, licquia_newman_2015_mw}. This is also true near the peak of the cosmic star formation history, $z=0-2$ \citep{tacconi_2013}, indicating that refueling is important across cosmic time. Furthermore, the empirical relationship between galaxies' star formation and gas content implies that the resupplying process is integral to galaxies' evolution through time \citep{kennicutt_1998}. This need for additional fuel is exacerbated by star-formation-driven outflows: winds from massive stars and supernovae are capable of launching outflows into galaxy haloes at rates greater than the SFR itself \citep{heckman_1990_outflows, rubin_2014_outflows, chisholm_18_outflows}, further depleting available gas reservoirs. Finally, cosmological simulations seem to demand inflows in order to reproduce galaxies' SFRs and buildup of stellar mass over cosmic time: the inflowing gas present in simulations can reach the disks of galaxies with $M^* < 10^{10.3} {\rm M_{\odot}}$ without being shock-heated, implying that it will be detectible as HI \citep{keres_2005_cold-accretion, dekel_birnboim_06_cold-accretion}.

While outflows from galaxy disks are well-studied phenomena in the extragalactic environment, inflows have proven more elusive to direct detection at the survey scale. High-column-density, inflowing structures such as high-velocity clouds (HVCs--\citealt{wakker_2004_hvc_bookchapter}) are ubiquitous in the Milky Way's immediate vicinity, and similar structures ought to exist in other haloes, as well. However, direct detection of HI in extragalactic HVC analogs is not practical, since the radio array configurations with sufficient spatial resolution lack the sensitivity to detect low-column-density structures. Furthermore, it is difficult to establish an unassailable link between gas in the vicinity of galaxies and the star formation in the galaxy itself: after all, cold gas is pervasive in galaxy haloes regardless of level of star formation activity \citep[e.g.,][]{bieging_78_etg-HI, sanders_80_etg-HI, zhang_2019_quiescent-HI}. Projection effects, unknown ionization state, obscuration on the far side of a dusty disk, and the paucity of metals all contribute to challenges in observing potential gaseous inflows. Targeted studies of individual galaxies have produced interesting examples of low-metallicity star formation indicative of ongoing accretion \citep{howk_2018_ngc4013-1, howk_2018_ngc4013-2, sanchez-almeida_2014}; but larger samples across a wider range of galaxy properties, conditions, and environments are needed. The difficult aspects of this open question must be addressed: to quantify the impact of inflowing gas is to ground one branch of the baryon cycle.

Since direct measurements of inflow are presently impractical at the survey scale, we must seek out its secondary effects, specifically on galaxies' chemistry: a galaxy's gas content is tied to the abundance of heavy elements in its interstellar medium (ISM), and therefore also to the buildup of stellar populations over cosmic time. Models of galaxy chemical evolution see gas mass fraction, gas-phase metallicity\footnote{Throughout this work, we use the terms ``oxygen abundance", ``gas-phase metallicity", and simply ``metallicity" interchangeably.}, and stellar mass as manifestations of galaxies' active transformation of gas into stars, with heavier elements released back into the ISM during the final stages of massive stars' lives \citep{tinsley_72, tinsley_73, vincenzo_2016_yields}. The observational evidence for this interplay is abundant in statistical samples of star-forming galaxies, with central metallicity increasing with stellar-mass \citep{tremonti_mz}, and gas fraction decreasing as stars and metals accumulate \citep{hughes_2013-gfrac-met}. It appears necessary to have some combination of gaseous inflows and outflows to explain the chemical abundances of old stars in the Milky Way \citep{spitoni_2019_mw-chemistry} and gas-phase abundances in other galaxies \citep{lilly_2013_gasreg}.

Because inflows may be brief and localized, rather than smooth in space and time, they may produce detectable chemical signatures in the disk. In general, a star-forming galaxy's metallicity decreases as galactocentric radius increases \citep{oey_kennicutt_1993, zaritsky_1994, sanchez_2014_metgrad, belfiore_2017_manga-metgrad, poetrodjojo_18, sanchez_19_review}. Reports differ regarding whether the slope of a galaxy's oxygen abundance profile (the metallicity gradient) correlates with its total stellar mass \citep{sanchez_2014_metgrad, belfiore_2017_manga-metgrad, zinchenko_2019, mingozzi_2020}. It has been argued that characteristic metallicity gradients emerge from inside-out galaxy formation \citep{prantzos_boissier_2000_metgrad}; but, they may simply emerge from the gaseous reservoirs' evolution \emph{at a local scale} \citep{moran_2012, zhu_2017_leakybox_smsd, barrera-ballesteros_2018, sanchez_almeida_localfmr, bluck_2019_global-local}. 

While common and statistically well-characterized on average, metallicity gradients are not perfectly uniform: chemical abundance does appear to vary azimuthally as well as radially, despite the rapid timescale (within one galactic rotation period) on which metals ejected from massive stars are thought to mix with the surrounding ISM \citep{petit_2015_azimixing}. In the Milky Way, for instance, abundance gradients measured at different azimuth angles have been found to differ by $\pm 0.03 ~ {\rm dex ~ kpc^{-1}}$ \citep{balser_2015_MWmetgrad}; but it is not clear whether this behavior extends to all star-forming galaxies as a population \citep[for a diversity of views, see][]{kreckel_2016_interarm, sanchez-menguiano_2016, vogt_2017_metgrad-azi, ho_2017_metgrad-azi, ho_2018_metgrad-azi}. Recently, \citet{kreckel_2019_phangs_metgrads} have reported typical metallicity dispersions of $0.03 - 0.05 ~ {\rm dex}$ at fixed radius in a sample of galaxies observed with VLT/MUSE. Furthermore, deviations from single gradients in individual galaxies have been observed---albeit in somewhat smaller samples---with breaks in the radial metallicity profiles separating the disks' innermost regions, intermediate radii, and outskirts \citep{sanchez-menguiano_2019}. Simultaneously, integral-field surveys such as MaNGA have yielded reports of anomalously-low-metallicity (ALM) regions, atypically-metal-poor areas at $\sim {\rm kpc}$ scales: \citet{hwang_2019_manga_almrs} finds a sizable ALM sample (defined as having oxygen abundance at least $0.14 ~ {\rm dex}$ below the mean metallicity for all MaNGA spaxels at the same stellar mass surface density and total galaxy stellar mass) in the MaNGA sample. About 25\% of local star-forming galaxies reportedly exhibit these characteristics, preferring galaxies below $10^{10} {\rm M_{\odot}}$ and morphologically-disturbed galaxies; and about 10\% of all MaNGA star-forming spaxels have the ``ALM" designation. The suggested explanation is the rapid, impulsive (``bursty") accretion of gas from the halo which fuels star formation.

Despite the recent evidence that has emerged for localized inflows onto star-forming disks, the details remain indistinct. Cosmic filaments, the source for inflowing gas in cosmological simulations, are many times the size of galaxies' star-forming disks \citep{martin_inflows}; and some questions remain about whether their associated inflows could effect a sustained and detectible depression of metallicity. The duty cycle of inflows (i.e., the fraction of time that the average gas reservoir actively accretes gas) is also relatively ill-constrained. However, to establish the definitive link between ALM gas and the actual inflow, the local gas reservoir should be characterized---an important deficiency in the current generation of observations. At present, the best indication of local gas supplies at the survey scale is global (single-dish) HI measurements. In addition, given the relatively short mixing timescale of heavy elements in the ISM, the width of the metallicity distribution function at fixed radius (or within a narrow radial range) could be employed as a tracer of newly-introduced gas with non-ambient metallicity. This work will address such opportunities.

This study seeks to link the diversity of radial metallicity profiles \& the width of the metallicity distribution function at fixed radius with galaxies' atomic gas reservoirs, by exploring the mutual correlations between radial oxygen abundance slope, oxygen abundance dispersion, total HI mass, and total stellar mass; and employs a conceptual model of local dilution to estimate plausible enhancements to star-forming gas reservoirs at the local (kpc) scale. In Section \ref{sec:data}, we describe the MaNGA resolved spectroscopy and three related value-added catalogs (VACs) which provide measurements of total galaxy stellar-mass, measurements of disk effective radius, and total HI mass. In Section \ref{sec:abund_meas_grad_dec}, we describe the strong-line metallicity calibration used; measure radial metallicity profiles with a metallicity decrement (a replacement for the more traditional gradient); measure azimuthal metallicity variations using the width of the metallicity distribution in a narrow annulus; and delineate the sample selection. In Section \ref{sec:trends}, we report the trends between HI mass, metallicity decrement, metallicity distribution width, and total stellar mass. In Section \ref{sec:modeling}, we describe the local dilution of star-forming gas reservoirs with an intuitive model. In Section \ref{sec:disc}, we summarize our results \& their implications. Throughout this work, we adopt the nine-year WMAP cosmology \citep{wmap9}, and a \citet{kroupa_imf_01} stellar initial mass function (IMF).