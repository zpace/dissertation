\section{Discussion and Conclusions}
\label{chap2:sec:disc}

In this work, we build upon \citetalias{pace_19a_pca} in this series, which used a library of 40000 spectra of composite stellar populations to construct a reduced-dimensionality space for fitting moderate signal-to-noise optical spectra of nearby galaxies; and as a complement to \citetalias{pace_19a_pca}, investigate the systematics of resolved stellar mass surface density (with respect to kinematically-measured dynamical mass surface density), total (aperture-corrected) galaxy stellar mass (with respect to photometric estimates), and flux-weighted stellar mass-to-light ratios.

We use three \mplv galaxies also observed as part of the DiskMass Survey \citep{diskmass_i} as a testbed for evaluating the systematics of kinematic and spectroscopic estimates of dynamical and stellar mass surface-density. We find that the resolved, inclination-corrected stellar mass surface-density obtained by a simple transformation between PCA-derived $\Upsilon^*_i$ and $M^*$ in most cases exceeds the estimates of dynamical mass surface density. Given that by definition, the dynamical mass includes stars, gas, and dark matter, this discrepancy indicates appreciable systematics in one or both methods. For the two higher-surface-brightness cases, better agreement between dynamical and SPS measurements could be obtained in two cases by postulating that the disk scale-height was overestimated by a factor of $\sim 1.5$. If, though, photometric mass estimates discussed in Section \ref{chap2:sec:photometry} were taken as truth (i.e., if the mass estimates here are anomalously heavy by that margin), the scale-height correction could be reduced to approximately a factor of 1.3.

\subsection{Galaxy total stellar-mass: aperture-correction and luminosity-weighting}

In order to estimate total galaxy stellar-mass (including that lying outside the grasp of the IFU), we test two rudimentary methods of aperture-correction: the ``ring" method applies a fiducial mass-to-light ratio equal to the median of the outermost-sampled $0.5 R_e$ to the difference between the NSA K-corrected $i$-band flux and the summed flux from the entire IFU; the favored ``CMLR" method uses the missing flux in the $g$ and $r$ bands to place the remainder of the galaxy on a CMLR. Furthermore, we note that the flux fraction lying outside the IFU is correlated with MaNGA subsample (on average, the Primary+ subsample is exposed out to a smaller galactocentric radius, so the aperture-correction is larger, and in the case of a negative gradient in stellar mass-to-light ratio, the difference between ``ring" and ``CMLR" corrections will be increased). We adopt the ``CMLR" method, since we believe it treats the Primary+ and Secondary samples more equally, and is less susceptible to over-correction.

We also examine the question of whether co-adding an entire IFU's spectra yields a different estimate of average stellar mass-to-light ratio. The intrinsic luminosity-weighting has been attributed to ``outshining" (where young, bright, blue spectra---having intrinsically low mass-to-light ratio---``wash out" the older spectra which sample most of the mass), as well as to the effects of dust. Both scenarios are simply a consequence of luminosity-weighting a mass-related property. We find that for MaNGA IFS data, galaxies with dust lanes or which are viewed edge-on experience the strongest mass-deficits between luminosity-weighted and IFU-summed stellar masses. It is plausible that a different IFS survey with different spatial-sampling characteristics or catalog selection function may experience a different balance of ``outshining" and dust-induced effects. This study is not alone in noting the deleterious effects of spatial binning on the robustness of spectral fits: in tests of the fidelity of full-spectral fitting, \citet{ibarra-medel_avila-reese_19} find that synthetic observations of hydrodynamic simulations of Milky Way-like galaxies with a MaNGA-like instrument, under MaNGA-like observing conditions, and with a spatial binning scheme (intended to increase signal-to-noise ratio) can produce a total mass deficit of up to 0.15 dex. This deficit worsens as inclination increases and a galaxy becomes edge-on: in the extreme, at $i \sim 90^{\circ}$, the light is dominated by the stellar populations at the lowest line-of-sight-integrated optical depth---so, only the outermost ring of stars is seen.

Given the observation that spatial coadding produces biases at the 0.05--0.1 dex level on \logml{i}, it may be prudent to re-evaluate the circumstances under which spectra are automatically coadded. It has become common practice to add together spatially adjacent spectra having low signal-to-noise ratios using adaptive binning techniques like Voronoi binning \citep{cappellari_voronoi}, in order to achieve uncertainties better suited to stellar population analysis. By binning indiscriminate to the properties of the spectra themselves, though, one might group regions in a galaxy that are intrinsically very different. In the case of, for instance, a disk galaxy with star-forming spiral arms and an older bulge or thick disk (having respectively low and moderat-to-high stellar mass-to-light ratios), one spatial bin might include contributions from both a star-forming arm (intrinsically bright) and the nearby bulge or older, thick disk (generally dimmer), potentially biasing an estimate of \logml{i} low. A more reliable approach would preferentially continue to accumulate bins along paths where spectra are similar. One pathway to this might rely on principal component decomposition of observed spectra, followed by agglomeration of nearby spectra that are both nearby and similar in PC space.

\subsection{Public Data}

In addition to the soon-to-be-released resolved maps of \logml{i} described in \citetalias{pace_19a_pca}, we will also release in tabular format estimates of total galaxy stellar-mass as a value-added catalog (VAC) in SDSS Data Release 16 (DR16). Included will be IFU-coadded stellar masses, stellar masses interior to 1 and 2 $R_e$ (where appropriate), and aperture-corrections using the ``ring" and (recommended) ``CMLR" methods described above.