\chapter[Applications of PCA-based stellar mass estimates]{Resolved and Integrated Stellar Masses in the SDSS-IV/MaNGA Survey, Paper II: Applications of PCA-based stellar mass estimates}
\label{chapter2}

\vfill

\begin{flushright}
    \fixspacing % Single spacing
    \textit{A version of this chapter has previously appeared\\
        in the \emph{Astrophysical Journal}} \\ \vspace{1ex}
    Pace, et al.\ 2019, \apj, 883, 83 \\ \vspace{1ex}
    and is notated P19b throughout this dissertation
\end{flushright}

\vspace*{1in} % Leave a 1-in space at the bottom.

\cleardoublepage

%%%%%%%%%%%%%%%%%%%%%%%%%%%%%%%%%%%%%%%%%%%%%%%%%%%%%%%%%%%%% 
\begin{chabstract}
A galaxy's stellar mass is one of its most fundamental properties, but it remains challenging to measure reliably. With the advent of very large optical spectroscopic surveys, efficient methods that can make use of low signal-to-noise spectra are needed. With this in mind, we created a new software package for estimating effective stellar mass-to-light ratios $\Upsilon^*$ that uses principal component analysis (PCA) basis set to optimize the comparison between observed spectra and a large library of stellar population synthesis models. In Pace et al. (2019a), we showed that a with a set of six PCA basis vectors we could faithfully represent most optical spectra from the Mapping Nearby Galaxies at APO (MaNGA) survey; and we tested the accuracy of our M/L estimates using synthetic spectra. Here, we explore sources of systematic error in our mass measurements by comparing our new measurements to data  from the literature. We compare our stellar mass surface density estimates to kinematics-derived dynamical mass surface density measurements from the DiskMass Survey and find some tension between the two which could be resolved if the disk scale-heights used in the kinematic analysis were overestimated by a factor of $\sim1.5$. We formulate an aperture-corrected stellar mass catalog for the MaNGA survey, and compare to previous stellar mass estimates based on multi-band optical photometry, finding typical discrepancies of 0.1 dex. Using the spatially resolved MaNGA data, we evaluate the impact of estimating total stellar masses from spatially unresolved spectra, and we explore how the biases that result from unresolved spectra depend upon the galaxy’s dust extinction and star formation rate. Finally, we describe a SDSS Value-Added Catalog which will include both spatially resolved and total (aperture-corrected) stellar masses for MaNGA galaxies.
\end{chabstract}
\cleardoublepage