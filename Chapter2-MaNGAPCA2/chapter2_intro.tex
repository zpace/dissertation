\section{Introduction}
\label{chap2:sec:intro}

A galaxy's total stellar mass is a helpful indicator of its overall evolutionary state: more massive galaxies tend to reside in older, more massive dark matter haloes \citep{gallazzi_charlot_05, gallazzi_06}, and they tend to have exhausted or expelled the majority of their cold gas in previous generations of star-formation star formation \citep{kauffmann_heckman_white_03, balogh_04_cmd, baldry_06_massquenching}. Similarly, more massive galaxies have stellar populations and gas which are relatively metal-enriched \citep{gallazzi_charlot_05, gallazzi_06}. In contrast, low-mass galaxies are still forming stars from their high mass fraction of cold gas \citep{mcgaugh_de-blok_97}, and continue to enrich their interstellar medium (ISM) from a relatively pristine chemical state \citep{tremonti_mz}.

It is possible to roughly quantify a galaxy's total stellar mass by measuring its stellar mass-to-light ratio $\Upsilon^*$ and then multiplying by the galaxy's luminosity. Two main classes of methods have been employed to make this calculation. The DiskMass Survey \citep[DMS, ][]{diskmass_i}, for instance, measured stellar and gas kinematics for 30 face-on disk galaxies in the local universe, and combined them with estimates of typical disk scale-heights to calculate dynamical mass surface density in several radial bins for each galaxy. Such dynamical measurements are limited, though, by potential systematics in the adopted disk scale-heights and assumptions about the stellar velocity dispersions \citep{aniyan_freeman_16, aniyan_freeman_18}.

An alternative is to compare the light emitted by stars in a galaxy to stellar population synthesis models \citep{tinsley_72, tinsley_73}. This involves wedding theory of stellar evolution to a stellar initial mass function (IMF), a star-formation history, and either theoretical model stellar atmospheres or empirical libraries of stellar photometry or spectroscopy. In general, one can either attempt to match an observed spectral-energy distribution (SED) directly to a star-formation histories (SFH), or use a library of SFHs to quantify, for example, a relationship between an optical color and a stellar mass-to-light ratio \citet{bell_dejong_01, bell_03}. Related approaches for optical spectra \citep{kauffmann_heckman_white_03} rely on measuring stellar spectral indices, such as the 4000-\mbox{\AA} break (\Dn) or the equivalent-width of the H-$\delta$ absorption feature (\HdeltaA). The process of reconstructing the exact SFH that produces an optical spectrum, though, is fraught with degeneracies: broadly speaking, mean stellar age, stellar metallicity, and attenuation due to dust are extremely covariate. Thus, recent efforts have concentrated on reducing the intractability of this problem.

In \citet[][hereafter \citetalias{pace_19a_pca}]{pace_19a_pca}, we applied a relatively innovative approach to inferring a stellar mass-to-light ratio from an optical spectrum, which used a library of $\sim$ 40000 SFHs and the associated optical spectra to construct a more computationally-friendly fitting framework, following the method of \citet[][, hereafter \citetalias{chen_pca}]{chen_pca}. Using principal component analysis (PCA), we computed a set of six basis vectors which were able to faithfully represent most optical spectra from the MaNGA survey\footnote{These six ``eigenspectra" do not themselves directly represent physical quantities; rather, when combined, they can serve to emulate spectra of stellar populations, and so are taken to encode more abstractly quantities of interest (such as stellar mass-to-light ratio, stellar metallicity, or dust attenuation).}. By projecting each observed spectrum down on the basis set of the principal component vectors, and adopting covariate observational uncertainties, we better account for theoretical degeneracies in stellar population synthesis, and obtain estimates of stellar mass-to-light ratio usable at a variety of signal-to-noise ratios, metallicities, and foreground dust attenuations.

Galaxies are not single points of light, though. Stellar populations vary on spatial scales comparable to the sizes of individual H\textsc{ii} regions, and undertaking observations that sample only at kiloparsec (or coarser) resolution therefore mix together light from locations in a galaxy that have very different properties and physical conditions. Thus, some information will necessarily be lost. For example, several studies have found that measurements based on integrated color-mass-to-light relations (CMLRs) tend to underestimate the total stellar mass (by 25--40\%, depending on specific star-formation rate) because dusty regions contribute very little to the integrated colors \citep{zibetti_2009,sorba_sawicki_15,martinez-garcia_17}. Explanations for this phenomenon run the gamut from simple variation in intrinsic luminosity of stellar populations across a galaxy's face (the ``outshining hypothesis") to the spatial variation of dust inducing a simultaneous reduction in total luminosity and increase in effective stellar mass-to-light ratio. Furthermore, the same mechanism need not dominate when transforming between different spatial scales (sub-kiloparsec to several-kiloparsec, or kiloparsec to galaxy-wide).

In this work, we build further on the resolved estimates of stellar mass-to-light ratio and resolved estimates of stellar mass obtained in \citetalias{pace_19a_pca}: In Section \ref{chap2:sec:data}, we briefly summarize the SDSS-IV/MaNGA survey and the data products upon which our analysis depends. In Section \ref{chap2:sec:dms_compare}, we compare PCA-derived estimates of resolved stellar mass surface-density (SMSD) to those of dynamical mass surface density (DMSD) from the DiskMass Survey. In Section \ref{chap2:sec:mstar_catalog}, we obtain a catalog of total galaxy stellar mass after testing two aperture-correction methods designed to account for stellar mass outside the spatial sampling area of the MaNGA instrument. In Section \ref{chap2:subsec:resolution_effects}, we examine the effects of spatially-coadding all spectral pixels on the total galaxy stellar mass obtained. Finally, in Section \ref{chap2:sec:disc}, we briefly summarize our results and outline the future release of a SDSS value-added catalog (VAC) of total galaxy stellar-masses.