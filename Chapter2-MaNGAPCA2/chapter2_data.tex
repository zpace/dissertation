\section{Data}
\label{chap2:sec:data}

This work further analyzes principal component analysis (PCA) fits from \citet{pace_19a_pca} to integral-field spectroscopic data from the MaNGA survey \citep{bundy15_manga}, part of SDSS-IV \citep{blanton_17_sdss-iv}. MaNGA is an integral-field survey which targets upwards of 10,000 nearby galaxies ($0.01 < z < 0.15$). The NASA-Sloan Atlas \citep[NSA, ][]{blanton_11_nsa} provides the majority of the targets for the MaNGA survey. Two-thirds of targets are drawn from the ``Primary+" sample, which have spatial coverage to at least 1.5 $R_e$; and the remaining one-third from the ``Secondary" sample, which have spatial coverage to 2.5 $R_e$. In order to obtain an approximately-flat distribution in galaxy $\log M^*$ \citep{manga_sample_wake_17}, MaNGA targets are selected uniformly in $i$-band absolute magnitude \citep{fukugita_96_sdss_photo, doi_2010_sdssresponse}. Within a particular redshift range, the MaNGA sample is also selected to be volume-limited. Absolute magnitudes, tabulated in the \texttt{DRPALL} catalog file, have been calculated using K-corrections computed with the \texttt{kcorrect v4\_2} software package \citep{blanton_roweis_07}, which assumed a \citet{chabrier03} stellar initial mass function and \citet{BC03} SSPs.

MaNGA observations employ the BOSS spectrograph \citep{smee_boss_instrument, sdss_boss_dawson_13}, an instrument on the SDSS 2.5-meter telescope at Apache Point Observatory \citep{gunn_sdss_telescope}. The spectrograph covers the optical--near-IR wavelength range (3600 to 10300 $\mbox{\AA}$) at a spectral resolution $R \sim 2000$. Galaxies are spatially-sampled by coupling the BOSS spectrograph's fiber feed to closely-packed hexabundles of fiber-optic cables, called integral-field units (IFUs), which each have between 19 and 127 fibers \citep{manga_inst}. Each fiber subtends 2" on the sky. Like previous SDSS surveys, hexabundles are affixed to the focal plane using a plugplate, and are exposed simultaneously \citep{sdss_summary}. Sky subtraction relies on 92 single fibers spread across the focal plane. Twelve seven-fiber ``mini-bundles" (six per half-plate) simultaneously observe standard stars, and are used for spectrophotometric flux calibration \citep{manga_spectrophot}.

MaNGA data are provided in both row-stacked spectra (RSS) and datacube (LINCUBE \& LOGCUBE) formats. RSS exposures are rectified into a datacube using a modified Shepard's algorithm, such that the size of the spatial element (spaxel) is 0.5" by 0.5" \citep{manga_drp}. The LOGCUBE products have logarithmic wavelength spacing ($d\log \lambda = 10^{-4}$, $d\ln \lambda \approx 2.3 \times 10^{-4}$)\footnote{In this work, the notation $\log$ denotes a base-10 logarithm, and $\ln$ denotes a base-$e$ logarithm.}. The dithering approach ensures that 99\% of the face of the target object is exposed to within 1.2 \% of the target depth \citep{manga_obs}. Sets of three exposures are accumulated until a threshold signal to noise ratio is achieved \citep{manga_progress_yan_16}. The typical point-spread function of a MaNGA datacube has a FWHM of 2.5" \citep{manga_obs}. The MaNGA Data Analysis Pipeline \citep[DAP, ][]{manga_dap} measures stellar kinematics, emission-line strengths, and stellar spectral indices for individual spaxels. The PCA analysis undertaken in \citetalias[][, the results of which we employ here]{pace_19a_pca} relies on the DAP products to establish a velocity field and deredshift individual spectra into the rest-frame.

This work builds on \citetalias{pace_19a_pca}, which inferred stellar mass-to-light ratio using a PCA-based spectral-fitting and parameter estimation method similar to earlier work from the SDSS-III/BOSS survey \citepalias{chen_pca}. This analysis relied upon a set of composite stellar populations (CSPs), whose optical spectra were used to construct a lower-dimensional spectral-fitting basis set. The spectral-fitting paradigm adopted in \citetalias{pace_19a_pca} takes into better account uncertainties in stellar population synthesis (such as the age-metallicity degeneracy), \cite2{along with covariate uncertainty accounting for imperfect spectrophotometry}. \citetalias{pace_19a_pca} analyzed \nrungalaxies galaxies drawn randomly from \mplvfull (\mplv), a set of \mplvngal observations of galaxies taken between March 2014 and June 2018, and returned resolved estimates \& their uncertainties for quantities of interest such as $i$-band effective stellar mass-to-light ratio, \logml{i}. \citetalias{pace_19a_pca} established the quality of the \logml{i} estimates by vetting them against synthetic observations of held-out test models: PCA-based estimates of \logml{i} were found to have on average very modest ($\lesssim 0.1~{\rm dex}$) systematics over a wide range of optical colors, stellar metallicites, and foreground dust attenuations. The estimates of uncertainty for \logml{i} were also found to be realistic.