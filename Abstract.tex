\chapter*{Abstract}
\addcontentsline{toc}{section}{Abstract}

Throughout their lives, galaxies form stars from their supply of cold gas: the largest of those stars generate heavy elements in their interiors prior to their explosive demise. As the heavy elements created in stellar evolution continually build with each generation of star formation, so do the generations of long-lived, low-mass stars. A galaxy's metal content and its total mass in stars together indicate the state of the galaxy's underlying gas reservoir: the reservoir is depleted by star formation and feedback, and is thought to be rejuvenated by inflows of low-metallicity gas from filaments of the cosmic web. However, the buildup of stellar mass is difficult to measure precisely: the observational degeneracies between stars with different ages and metallicities bring about troublesome systematics. In addition, there is little direct evidence for inflows' importance in the local universe, though they are present in simulations and seem to be necessary to maintain gas reservoirs' star-forming vigor. 

In this dissertation, I develop and refine a method of measuring stellar mass-to-light ratio and other stellar population properties from medium-resolution optical spectroscopy. This method builds on a library of model star-formation histories and their associated synthetic optical spectra, and constructs a low-dimensional spectroscopic basis set capable of maximizing the predictive power of observations. This method is tested and deployed on resolved, integral-field spectroscopic observations from the SDSS-IV/MaNGA survey of nearly 10,000 nearby galaxies. Finally, I produce and release a catalog of resolved stellar mass maps and of aperture-corrected total galaxy stellar masses.

I also measure resolved gas-phase metallicities in the MaNGA survey, and relate them to the mass of the galaxy-wide gas reservoir. A mutual correlation is uncovered between a steep radial metallicity profile, a large dispersion in the metallicity profile between $1.25 - 1.75 ~ R_e$, and a large HI mass fraction relative to galaxies of the same total stellar mass. The first axis of that correlation is consistent with theoretical predictions of the signatures radial gas flows, so I test a simple, but intuitive model of a gaseous inflow, whereby ambient metallicity is ``diluted '' by low-metallicity gas introduced from elsewhere. This yields estimates of the possible impact of gaseous inflows on local star-forming gas reservoirs; and indicates a means towards selecting potential inflow hosts for radio follow-up.
