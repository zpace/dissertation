\chapter{Discussion, Summary, and Conclusions}

The chemical evolution of galaxies arises due to a confluence of dark matter halo collapse, star-formation, gas expulsion, and likely rejuvenation of star-forming gaseous reservoirs. The result is exquisite correlations between properties of halos, the galaxies embedded within them, and even individual regions within those galaxies. The works above have sought to measure the properties (particularly the mass) of the stars, the quantity of metals in the ISM, and their mutual relationship with the gaseous reservoir.

\section{Measuring stellar masses: resolved and total}

In Chapter \ref{chapter1}, a novel method for inferring the stellar mass-to-light ratio from moderate-resolution optical spectra is introduced and deployed. This method relies on a pre-generated suite of star-formation histories (SFHs, also called ``training data''): their optical spectra are used to generate a low-dimensional spectroscopic basis set (``eigenspectra''), which aid in evaluating the goodness-of-fit for each model SFH against each observed spectrum. This is a relatively cheap means of accounting for observational degeneracies, such as between stellar age, metallicity, and foreground dust attenuation. This method also permits estimation of properties of observed spectra such as stellar mass-to-light ratio, provided they are known for the training data; and performs well when tested on held-out synthetic data across a wide signal-to-noise range ($2 \le S/N \le 20$). Maps of stellar mass-to-light ratio are generated for the full available MaNGA dataset (as of June 2019), all of which are publicly available.

Chapter \ref{chapter2} investigates systematics of the stellar mass measurements made in Chapter 1, and aggregates the resolved measurements into total galaxy stellar mass estimates. By comparing the stellar mass surface density (SMSD) of three moderately-inclined MaNGA disk galaxies to corresponding measurements of dynamical mass surface density (DMSD) from the DiskMass Survey, we identify an approximately-factor-of-two excess of SMSD with respect to DMSD: for the two galaxies with higher surface brightness, the discrepancy could be resolved by assuming a factor of $\sim 1.5$ overestimate of the disk scale height $h_z$ systematically depressed the DMSD measurements. This is concievable given its estimation from scaling relations; and corresponds to predicted consequences of tracking disk kinematics and photometric properties with different types of stars. 

Next, the maps of resolved stellar mass are summed and aperture-corrected, creating a catalog of total galaxy stellar mass. We test two methods of aperture-correction, and recommend one which uses a color--mass-to-light ratio relation (CMLR) to estimate the stellar mass-to-light ratio of unobserved galaxy regions. We finally investigate whether co-adding an entire galaxy's spectrum (i.e., obliterating positional information and weighting resolved mass-to-light ratio by surface brightness) yields a different estimate of total stellar mass than coadding all spaxels' resolved masses. We find that galaxies with unequal dust attenuation and galaxies with high concentrations of central star-formation (as measured by the $\rm H\alpha$ emission line) have the greatest degree of disagreement between brightness-weighted and IFU-summed total stellar masses. The factor-of-several systematics in the worst case speaks to the deleterious effects of summing spectra with different brightnesses and underlying stellar populations.

\section{Anomalous metallicity profiles: a signal of inflows?}

Chapter \ref{chapter3} relates galaxies' integrated HI content with their gas-phase metallicity profiles. After identifying a sample of galaxies having robust metallicity profiles along their major axis, we measure their metallicity decrement between two radial annuli: $0.0 - 0.5 ~ R_e$ \& $1.25 - 1.75 ~ R_e$. We also measure the scatter of the metallicity distribution in the outer radial annulus. By merging those measurements with single-dish measurements of total HI mass, we identify a mutual correlation between a large metallicity decrement, large metallicity scatter in the outer annulus, and HI mass fraction. For galaxies with stellar mass less than $10^{10.7} ~ M_{\odot}$, the correlation is statistically significant to the 5\% level, and does not correspond to changes in total stellar mass or central oxygen abundance. 

We next test one possible interpretation of this correlation, that a low-metallicity gaseous inflow has diluted the metallicities in some areas of galaxy outskirts, disturbing a ``normal'' metallicity gradient. We find that the outer annuli of $\sim 25\%$ of galaxies have metallicities are consistent with dilution by the addition of $10\%-90\%$ of the local reservoir's mass in low-metallicity gas. This is once again the case for galaxies with stellar mass less than $10^{10.7} ~ M_{\odot}$. Probable inflow hosts (defined as having an anomalously-high metallicity scatter in the interval $1.25 - 1.75 ~ R_e$) have global HI mass enhancements of $0.05 - 0.25 ~ {\rm dex}$ relative to peer galaxies at the same mass but lower metallicity scatter. Assuming an inflow covering fraction of 50\% (broadly consistent with the observation of a broadened metallicity distribution at fixed radius, rather than simply a steepened metallicity profile that would emerge from a 100\% covering fraction), the observed enhancements to the inflow host candidates' total HI mass exceed that implied by gas-enhanced reservoirs at $1.25 - 1.75 ~ R_e$. That is, the sum of the local reservoirs' inflow-induced HI enhancements is less than the observed global HI enhancements.

The mismatch between implied local and observed global HI enhancements implies that if an inflow is producing the observed chemical signatures, there should be a significant amount of gas present at large radii, possibly pointing to active accretion from the cosmic web. We suggest that IFS galaxy surveys are well-equipped to spot secondary signals of gas inflow (such as steepened metallicity profile \& increased scatter of the metallicity distribution at fixed radius). Along with enhanced global HI mass, these cues could aid in selecting probable inflow candidates for follow-up with the VLA or other high-resolution radio facilities.

\section{Future direction}

Here are briefly summarized some of this dissertations results, and discussions of how they indicate potential future developments in the field.

\begin{itemize}
    \item The adoption of a synthetic stellar library for stellar population synthesis is believed to be the largest source of systematics in the resolved measurements of stellar mass-to-light ratio (Chapter \ref{chapter1}). At this time, there exists no empirical stellar library with the correct wavelength coverage \& spectral resolving power, as well as acceptable coverage of the color-magnitude diagram (age metallicity, $\frac{\rm \alpha}{\rm Fe}$, TP-AGB stage, etc.). The MaStar project \citep{yan_19_mastar} has sought to fill this void, and the 3321 unique stellar spectra have recently been integrated into population synthesis models \citep{maraston_20_mastar-models}, and will likely sharply reduce systematics for stellar ages less than 0.2 Gyr, especially when paired with low-dimensional techniques such as PCA.
    \item Estimating a single mass-to-light ratio for a galaxy ignores the vastly different conditions within: to avoid typical biases of $0.05 - 0.1 ~ {\rm dex}$ (and worst-case biases greater than $0.3 ~ {\rm dex}$), spectra should not be coadded indiscriminately. Adaptive binning techniques such as Voronoi binning should seek to coadd only nearby spectra that are similar.
    \item Galaxies' radial metallciity profiles and their azimuthal variations have been shown to be suggestive of ongoing, impulsive accretion of low-metallicity gas. With massively-multiplexed IFS galaxy surveys, it may be possible to select probable inflow hosts based on their chemistry (and neutral gas content), and follow up promising candidates in the radio to directly detect inflowing gas.
    \item Additional, high-resolution radio follow-up of galaxies with IFS observations will aid in calibrating indirect measures of gas content. If attenuation and metallicity can be shown to provide acceptable estimates of the local gas mass, it will become much easier to probe stellar populations' effects on their nearby ISM.
\end{itemize}