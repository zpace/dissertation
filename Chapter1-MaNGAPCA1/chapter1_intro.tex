\section{Introduction}
\label{chap1:sec:intro}

A galaxy's stellar mass is one of its most important physical properties, reflecting its current evolutionary state and future pathway. On the whole, more massive systems tend to possess older stellar populations \citep{gallazzi_charlot_05, gallazzi_06} with very little current star formation \citep{kauffmann_heckman_white_03, balogh_04_cmd, baldry_06_massquenching}, a small gas mass fraction \citep{mcgaugh_de-blok_97}, higher gas-phase metallicity \citep{tremonti_mz}, and stellar populations enhanced in $\alpha$-elements relative to iron \citep{thomas_maraston_korn_04, thomas_05}. Fundamentally, a galaxy's stellar mass indicates the total mass of the dark matter halo in which it is embedded \citep{yang_03_smhmr, behroozi_wechsler_conroy_13, somerville_behroozi_2018}: the higher the mass of the dark matter halo, the more evolved the galaxy tends to be, and the lesser the galaxy's capacity for future star formation.

Traditionally, two methods have been used to estimate galaxy stellar mass: kinematics and stellar population analysis. By measuring the average motions of stars, the dynamical mass (a distinct but related property which includes both baryonic and dark matter), can be determined. The DiskMass Survey \citep[DMS,][]{diskmass_i} used measurements of the vertical stellar and gas velocity field and stellar velocity dispersion $\sigma^*_z$, in concert with inferred values of disk scale height $h_z$ to estimate the azimuthally-averaged dynamical mass surface density $\Sigma^{\rm dyn}$ of 30 local, low-inclination disk galaxies within several radial bins \citep{diskmass_vii}. However, dynamical measurements are subject to systematics related to the vertical distribution and scale height of stars, how the vertical velocity is measured \citep{aniyan_freeman_16, aniyan_freeman_18}, and the typical assumption of a constant stellar mass-to-light ratio used in Jeans-based estimates \citep{bernardi_sheth_17}.

The second method of stellar mass estimation relies on comparing photometry or spectroscopy of galaxies to stellar population synthesis (SPS) models. SPS weds theoretical stellar isochrones to theoretical model atmospheres or observed libraries of stellar spectra, under the constraint of the stellar initial mass function (IMF), in order to obtain an estimate of the mass-to-light ratio, and therefore, the mass. \citet{tinsley_72,tinsley_73} defined the fundamentals of this method, combining the analytic expressions for the stellar IMF, star formation rates (SFR), and theory of chemical enrichment. \citet{bell_dejong_01} and \citet{bell_03} later took existing stellar models and described empirical relationships between optical colors and stellar mass-to-light ratios. Other approaches infer a star formation history (SFH) from broadband, multi-wavelength spectral energy distributions (SEDs): in such a case, the starlight itself can be observed in many bands \citep{shapley_05_seds}, or its indirect consequences can also be considered, such as infrared dust emission that arises after stars form \citep{dale_01_infrared_dust}. Software libraries such as \texttt{MagPhys} \citep{da-cunha_charlot_elbaz_08,da-cunha_charlot_11_magphys}, \texttt{Cigale} \citep{burgarella_05, giovannoli_11, serra_11}, and \texttt{Prospector} \citep{leja_johnson_conroy_van-dokkum_17_prospector} take this approach, often (but not always) after adopting a family of SFHs. In short, estimates of stellar mass-to-light are generally made by finding the combination of simple stellar populations (SSPs; i.e., stars of a single age and metallicity) that produces the best match to an observed galaxy spectrum or photometry.

Simple SFH scenarios, such as \citet{bell_03}, produce almost-linear relationships (often referred to as color-mass-to-light relations, or CMLRs) between optical color and the logarithm of stellar mass-to-light ratio. This can be a convenient first tool, but there are significant systematics associated with stellar IMF, metallicity, and attenuation by dust (see Section \ref{chap1:subsec:cmlrs}). Often, different CMLRs produce extremely contradictory mass-to-light estimates \citep{mcgaugh_schombert_schombert_14}. We demonstrate below that inferring stellar mass-to-light ratio from optical spectra offers some improvements over CMLRs.

Additionally, certain spectroscopic features---such as the strength of the 4000$\mbox{\AA}$ break \cite[\Dn:][]{bruzual_83, balogh_99, balogh_00}, equivalent width of the H$\delta$ absorption line \citep[\HdeltaA:][]{worthey_ottaviani_97}, and several other atomic and molecular indices \citep[e.g. CN, Mg$b$, NaD:][]{worthey_94}---have been used to estimate mean stellar age, metallicity, activity of recent starbursts, and stellar mass-to-light \citep{kauffmann_heckman_white_03,gallazzi_charlot_05,silchenko_06,wild_kauffmann_07_pca}. Spectral indices are akin to optical colors in that they are a lower-dimensional view on a galaxy's spectrum, but a view designed to effectively capture an informative phase of stellar evolution.

The advent of large spectroscopic surveys with good spectrophotometric calibration has enabled more widespread use of full-spectral fitting: spectra spanning a large fraction of the visible wavelength range offer a much more detailed view on a galaxy's SED, albeit within a smaller overall wavelength range than techniques which simultaneously examine UV, optical, infrared, and radio domains. Many software libraries exist for performing such analysis, including such as \texttt{FIREFLY} \citep{firefly_wilkinson_pmanga}, \texttt{STECKMAP} \citep{steckmap}, \texttt{VESPA} \citep{vespa_tojeiro}, \texttt{pPXF} \citep{cappellari_ppxf}, \texttt{STARLIGHT} \citep{starlight}, and \texttt{Pipe3D} \citep{pipe3d_i, pipe3d_ii}. Very recent developments include techniques which simultaneously consider spectroscopic and photometric measurements \citep{chevallard_charlot_16_beagle, thomas_le-fevre_17, fossati_mendel_18}.

The reliability of the resulting spectral fits is hampered by four main factors. First, certain phases of stellar evolution, such as the thermally-pulsating asymptotic giant branch (TP-AGB) stage, are still poorly understood, and this causes troublesome systematics \citep{maraston_06, marigo_08}. Second, due to the degeneracy between stellar population age and metallicity, it is difficult to map spectral features uniquely to a combination of stellar populations. Modern spectroscopic surveys alleviate this somewhat with the inclusion of the NIR Ca\textsc{ii} triplet \citep{terlevich_caII_89, vazdekis_caII_03}, a feature that is sensitive to Calcium abundance (and secondarily, overall metal abundance) in stars older than 2 Gyr \citep{usher_beckwith_18_CaT}, as well as other spectral indices \citep{spiniello_trager_12, spiniello_trager_14}, but care is still required. Third, the process of stellar population-synthesis relies on fully-populating the parameter space of temperature, surface gravity, metal abundance, and ${\rm [\frac{\alpha}{Fe}] }$ (usually combining multiple stellar libraries, interpolating across un-sampled regions of parameter space, or patching with theoretical libraries). Fourth, it is unclear how to best recover the information contained in spectra: some approaches continue to model spectra as the sum of simple mono-age, mono-abundance SSPs; but it is possible that the resulting numerical freedom may produce un-physical results.

This work's spectral-fitting technique follows \citet[][hereafter \citetalias{chen_pca}]{chen_pca}: in \citetalias{chen_pca}, principal component analysis (PCA) was performed on a ``training library" of synthetic optical spectra (the synthetic spectra were themselves produced using a stochastically-generated family of SFHs), yielding a set of orthogonal ``principal component" (PC) vectors. PCA is a method of finding structure in a high-dimensional point process \citep{jolliffe_1986_pca}, which has been applied in the field of astronomy to such problems as spectral-fitting \citep{budavari_09}, photometric redshifts \citep{cabanac_02}, and classification of quasars \citep{yip_connolly_04, suzuki_06}. PCA transforms data in many dimensions into fewer dimensions in a way that minimizes information loss. If a spectrum containing measurements of flux at $l$ wavelengths is interpreted as a single data point in $l$-dimensional space, a group of $n$ spectra forms a cloud of $n$ points. PCA finds the $q$ vectors (also called ``eigenvectors", ``eigenspectra", or ``PCs") that are best able to mimic the full, $l$-dimensional data. Equivalently, PCA finds a vector space in which the covariance matrix is diagonalized for those $n$ spectra.

In the PCA-based spectral-fitting paradigm, a set of 4--10 eigenspectra were used as a reduced basis set for fitting observed spectra at low-to-moderate signal-to-noise. That is, the best representation was found for each observed spectrum in terms of the eigenspectra. The goodness-of-fit was evaluated in PC space for each model spectrum in the training library: this was used as a weight in constructing a posterior PDF for stellar mass-to-light ratio. In this paradigm, the ``samples" of the PDF are simply the full set of CSP models, which have known stellar mass-to-light ratio. In other words, the training library both defines the eigenspectra and provides samples for a quantity of interest. This process can be carried out on thousands of observed spectra (using tens of thousands of models) simultaneously, and is Bayesian-like because the training library acts as a prior on the ``allowed" values of, for example, stellar mass-to-light ratio. \citet{tinker_mstar} utilized the stellar mass-to-light ratio estimates from \citetalias{chen_pca} to measure the intrinsic scatter of stellar mass at a fixed halo mass for high-mass BOSS galaxies, finding that PCA-based estimates of stellar mass correlated with total halo mass \emph{better} than photometric mass estimates, a result suggesting that the PCA estimates are more accurate.

Though this work uses a method very similar to \citetalias{chen_pca}, we also integrate new model spectra generated with modern isochrones and stellar atmospheres. We apply this method to galaxy spectra from SDSS-IV/MaNGA \citep[Mapping Nearby Galaxies at Apache Point Observatory,][]{bundy15_manga}, an integral-field spectroscopic (IFS) survey of 10,000 nearby galaxies ($z \lesssim 0.15$). The MaNGA survey is designed to enhance the current understanding of galaxy growth and self-regulation by observing galaxies with a wide variety of stellar masses, specific star formation rates, and environments. We produce resolved maps of stellar mass-to-light ratio for a significant fraction of the \mplvngal galaxies included in \mplvfull (\mplv).

The structure of this paper is as follows: in Section \ref{chap1:sec:data}, we discuss the MaNGA IFS data; in Section \ref{chap1:sec:SFHs}, we detail the procedural generation of star-formation histories and their optical spectra (the ``training data"), a \Dn--\HdeltaA comparison between the model library \& actual observations, and why CMLRs do not recover sufficient detail about the underlying stellar mass-to-light ratio; in Section \ref{chap1:sec:method}, we review the underlying mathematics of PCA, present its application to parameter estimation---concentrating, in particular, on why we might expect improvement over traditional methods in the case of IFS data---and examine the reliability of the resulting estimates of stellar mass-to-light ratio in relation to degenerate parameters like metallicity and attenuation; and in Section \ref{chap1:sec:discussion}, we provide example maps of stellar mass-to-light ratio for a selection of four late-type galaxies and three early-type galaxies, discussing possible future improvements to the spectral library, and outlining a future release of resolved stellar mass-to-light ratio maps. To complement this work's investigation of random errors in PCA-based estimates of stellar mass-to-light ratio, in \citet[][hereafter \citetalias{pace_19b_pca}]{pace_19b_pca}, we compare resolved maps of stellar mass surface density (derived from stellar mass-to-light ratio estimates made in this work) to estimates of dynamical mass surface density from the DiskMass Survey, in a view on the systematics of PCA-derived stellar mass-to-light ratios; and construct aperture-corrected, total stellar masses for a large sample of MaNGA galaxies.