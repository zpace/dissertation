\chapter[PCA spectral fitting \& stellar mass-to-light ratio estimates]{Resolved and Integrated Stellar Masses in the SDSS-IV/MaNGA Survey, Paper I: PCA spectral fitting \& stellar mass-to-light ratio estimates}
\label{chapter1}

\vfill

\begin{flushright}
    \fixspacing % Single spacing
    \textit{A version of this chapter has previously appeared\\
        in the \emph{Astrophysical Journal}} \\ \vspace{1ex}
    Pace, et al.\ 2019, \apj, 883, 82 \\ \vspace{1ex}
    and is notated P19a throughout this dissertation
\end{flushright}

\vspace*{1in} % Leave a 1-in space at the bottom.

\cleardoublepage

%%%%%%%%%%%%%%%%%%%%%%%%%%%%%%%%%%%%%%%%%%%%%%%%%%%%%%%%%%%%% 
\begin{chabstract}
We present a method of fitting optical spectra of galaxies using a basis set of six vectors obtained from principal component analysis (PCA) of a library of synthetic spectra of 40000 star formation histories (SFHs). Using this library, we provide estimates of resolved effective stellar mass-to-light ratio ($\Upsilon^*$) for thousands of galaxies from the SDSS-IV/MaNGA integral-field spectroscopic survey. Using a testing framework built on additional synthetic SFHs, we show that the estimates of \logml{i} are reliable (as are their uncertainties) at a variety of signal-to-noise ratios, stellar metallicities, and dust attenuation conditions. Finally, we describe the future release of the resolved stellar mass-to-light ratios as a SDSS-IV/MaNGA Value-Added Catalog (VAC) and provide a link to the software used to conduct this analysis\footnote{\label{footnote:software_link}The software can be found at \url{https://github.com/zpace/pcay}.}
\end{chabstract}
\cleardoublepage